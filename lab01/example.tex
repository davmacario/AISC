\documentclass[12pt]{article}
\usepackage[utf8]{inputenc}

\usepackage{hyperref}   % Change color and style of \ref
\hypersetup{
    colorlinks=true,
    linkcolor=blue,
    filecolor=magenta,      
    urlcolor=cyan,
    pdftitle={AISC 01},
    pdfpagemode=FullScreen,
}

\usepackage{graphicx} % Allows you to insert figures
\graphicspath{ {./images/} } % images are found in this position
\counterwithin{figure}{section} % Settings for figure numbering
\counterwithin{figure}{subsection}

\usepackage{amsmath} % Allows you to do equations
\newcommand{\Mod}[1]{\ (\mathrm{mod}\ #1)}
\usepackage{amsfonts} % Contains math. symbols fonts (e.g., "real numbers" set)
\usepackage{fancyhdr} % Formats the header
\usepackage{geometry} % Formats the paper size, orientation, and margins
\linespread{1.25} % about 1.5 spacing in Word
\setlength{\parindent}{0pt} % no paragraph indents
\setlength{\parskip}{1em} % paragraphs separated by one line

\usepackage{enumitem} % Used to reduce whitespace between list elements
\setlist[itemize]{noitemsep, topsep=0pt} % Set the whitespace above list to the minimum

\usepackage[format=plain,
            font=it]{caption} % Italicizes figure captions
\usepackage[english]{babel}
\usepackage{csquotes}
\renewcommand{\headrulewidth}{0pt}
\geometry{letterpaper, portrait, margin=1in}
\setlength{\headheight}{14.49998pt}

\newcommand\titleofdoc{Laboratory 1 - AISC} %%%%% Title
\newcommand\GroupName{Agnetta Stefano, Brozzo Doda Umberto, Macario Davide}
\newcommand\CurrDate{March 15\textsuperscript{th}, 2023}

\begin{document}
\begin{titlepage}
   \begin{center}
        \vspace*{4cm} % Adjust spacings to ensure the title page is generally filled with text

        \Huge{\titleofdoc} 

        \vspace{0.5cm}
        \LARGE{Breaking classical encryption schemes}
            
        \vspace{3 cm}
        \Large{\GroupName\\}
       
       
        \vspace{4 cm} % Optional additional info here
       
       
        \vspace{3 cm}
        \Large{\CurrDate}
        
        \vspace{0.25 cm}
        \Large{A.Y. 2021-2022}
       

       \vfill
    \end{center}
\end{titlepage}

\setcounter{page}{2}
\pagestyle{fancy}
\fancyhf{}
\rhead{\thepage}
\lhead{\GroupName; \titleofdoc}

\section{Introduction} % If you want numbered sections, remove the star after \section

To be filled

\section{Exercise 1 - Monoalphabetic cipher}



\section{Exercise 2 - Vigenère cipher}

The Vigenère is an encryption scheme based on the use of a key $k$ of length $m$ to encrypt the plaintext $p$ as follows:

\begin{equation*}
  c_i = E(p,k) = (p_i + k_{i\mod{m}})\mod{26}
\end{equation*}

This means that letters at distance $m$ are shifted by the same amount, as in a generic Caesar's cipher.
As a consequence, a possible approach for cryptanalysis can be that of looking at the ciphertext as a set of subsequences obtained by considering letters at distance $m$ of the cipher. 
Then, it is possible to act on these subsequences by analyzing letter frequencies. In this case, a brute force approach would be time consuming as the subsequences do not contain english words, but rather (pseudo-)random letters. 
In any case the frequency analysis is feasible, since these letters are extracted from an (encrypted) English text.

The cracking mechanism is based on evaluating the circular correlation between the letters in each subsequence and the theoretical vector of letter densities for the English language. 
By finding the maximum of the correlation it is possible to choose the key character for the current subsequence, as this point will correspond to the shift value which makes the letter frequencies of the subsequence match with the theoretical frequencies.
This, however requires to estimate the length of the key, and to achieve this another previous step is needed.

The estimate of the key length is done by extracting information from the ciphertext, once again revealing the main disadvantage of this encryption scheme.
Indeed, it is possible to evaluate a score, corresponding to the sum of squared frequencies of each letter for different subsequences lengths.
Each language has a specific value for this score, which, in the case of English is 0.065.

By comparing the scores as functions of $m$, it is possible to find out the best value for the key length as the one for which the score is closest to 0.065.

In the case of the ciphertext found in document `cryptogram02.txt', the estimated key length was 15 and the key was found to be ``nowyouseethekey''.

\subsection{Complexity of the algorithm}

The complexity of the algorithm for breaking the Vigenère cipher can be estimated as:

%

\section{Conclusions}


\pagebreak

\end{document}
