\documentclass[12pt]{article}
\usepackage[utf8]{inputenc}
\usepackage{fullpage}
\usepackage{graphicx}
\usepackage{url}
\usepackage{amsmath}
\usepackage{amssymb}
\usepackage{caption}
\usepackage{subcaption}
\usepackage{color,soul}
\usepackage{amsfonts}
\usepackage{wrapfig}
\usepackage{multicol}
\usepackage{hyperref}
\hypersetup{
   colorlinks=true,
   linkcolor=blue,
   filecolor=magenta,
   urlcolor=cyan,
   pdftitle={laboratory02–AISC},
   pdfpagemode=FullScreen
}
\graphicspath{ {./images/} }
\usepackage{fancyhdr}
\usepackage{enumitem}
\usepackage{epigraph}
\setlist[itemize]{noitemsep, topsep=0pt}
\setlist[enumerate]{noitemsep, topsep=0pt}
\title{Laboratory 2 – AISC}
\author{Agnetta Stefano, Brozzo Doda Umberto, Macario Davide}
\date{March 29\textsuperscript{th}, 2023}

\newcommand\titleofdoc{Laboratory 2 – AISC} %%%%% Title
\newcommand\GroupName{Agnetta Stefano, Brozzo Doda Umberto, Macario Davide}
\newcommand\CurrDate{March 29\textsuperscript{th}, 2023}

\begin{document}
\begin{titlepage}
   \begin{center}
        \vspace*{4cm} % Adjust spacings to ensure the title page is generally filled with text

        \Huge{\titleofdoc} 

        \vspace{0.5cm}
        \LARGE{A simplified version of AES}
            
        \vspace{3 cm}
        \Large{\GroupName\\}
       
        \vspace{4 cm}
       
        \vspace{4 cm}
        \Large{\CurrDate}
        
        \vspace{0.25 cm}
        \Large{A.Y. 2021–2022}
       
       \vfill
    \end{center}
\end{titlepage}

\section{Introduction}
\label{sec:01}

This laboratory revolved around the analysis of a simplified version of AES (Advanced Encryption Standard), which was provided as a python script, which can be found at \cite{Original Python implementation}, and which also contained all necessary sub–blocks for each round (SBox, ShiftRows, MixColumns and AddKey).
These blocks are designed to work on 16–bit inputs, organized as 2–by–2 matrices of 4–bit `\textit{nibbles}'. Also the key is a 16-bit string.

The focus of this analysis is to test the behavior of the simplified AES in terms of avalanche effect, and to cryptanalyze an improper implementation of a block cipher, based on the provided sub-blocks, in order to break a ciphertext.

\section{Avalanche effect}
\label{sec:02}

The goal of the first section of this laboratory was to demonstrate the avalanche effect of the simplified AES that follows a single bit flip after various manipulations of the algorithm, such as different number of rounds and the missing implementations of some fundamental steps.

To measure the avalanche effect the hamming distance will be computed: values around half the length of the elements in exam mean a good avalanche effect, since the correlation between them is at its lowest point.

Two different experiments will be taken.
In the first one, given a key, a plaintext is randomly generated and encrypted, then a bit of the plaintext is flipped and the result is encrypted. The key used is: {\color{blue}0b1111111111111111}.
In the second one, chosen a plaintext, the key is randomly generated and used to encrypt the plaintext; the same plaintext is then encrypted with the key with a single bit flipped. The used plaintext is: {\color{blue}0b1010101010101010}.
Both the experiments are repeated 1000 times, and the avalanche effect is measured referring to the average of the hamming distances of each single run.
Since they will be recalled more times, for the sake of convenience they will be respectively called "changed plaintext" and "changed key" experiments.

In the first step, the algorithm is composed of all its four fundamental blocks, which will be repeated two times. For the "changed plaintext" experiment the average hamming measured is around {\color{blue}4}, and a value of {\color{blue}8} is obtained for the "changed key".
The resulted diffusion property of the "changed key" is explained by the fact that when, during the second round, the key expansion is performed it changes radically the key itself, so when the xor is applied to the state it statistically will make this last one completely different from its starting value.
The other first experiment shows instead a weaker behavior: the state is treated as four blocks, and a change in one of them does not necessarily result in big differences after just two rounds of AES. Indeed, as the the number of rounds raises to 3, the avalanche effect measured starts to converge to values around {\color{blue}8} for both the experiments.
As the number of rounds raises the results continue to show a good average hamming distance, which sets to 3 number of minimum rounds needed to have a sufficient avalanche effect.

Now, for the next following step, referred as "lazy version", ShiftRows and MixColumns blocks are missing and as the experiments are done, the hamming distances decrease drastically for the "changed plaintext" experiment. This enlights how the diffusion property is strongly dependent on ShiftRows and MixColumns blocks. The reason lies in the way these two stages operate: they provide bit shuffling between different 4-bits nibbles of the state, and consequently if they're skipped the difference between two ciphertexts obtained from two plaintexts which have hamming distance equals to {\color{blue}1} cannot be more than 4, which is the size of a nibble.
Here two examples are reported.
***todo: table with two examples***
However this "lazy version" still produces an appreciable diffusion effect for the "changed key" experiment after a slight change in the key.

The last test, called "very lazy version", follows a cut on the key expansion block's behavior, so that only the first two round keys will be used for all the rounds. Finally the encryption becomes now ineffective, giving hamming distance values around {\color{blue}2} for both the experiments. Now a single bit change to the key in the "changed key" experiment will end in a less diffused result, since the key expansion gives now as a result the starting key value with the two bytes inverted in position.

These experiments showed how the AES algorithm needs to be fully implemented to grant diffusion and confusion properties in every cases, so that cryptanalysis techniques are almost impossible to be performed.


\section{Improperly implemented block cipher}
\label{sec:03}

The second part of this laboratory concerned the cryptanalysis of an improperly-implemented block cipher of which the description was given, and which used on the blocks that were provided for the simplified AES.

The proposed procedure was essentially that of carrying out a KPA having observed a single plaintext-ciphertext pair, which was obtained with the same key as the large ciphertext that needed to be broken.
The pair was the following:

\begin{itemize}
   \item \textbf{Known plaintext}: \verb|0b0111001001101110|
   \item \textbf{Known ciphertext}: \verb|0b0001111001100101| 
\end{itemize}

Having the possibility to observe the Python code which implemented the encryption (and decryption) mechanism, it was found that the block cipher only consisted of a single round, composed of ``Substitute Bytes'', ``Shift Row'' and ``Add Key'' (with a 16-bit subkey) blocks only, whose order was reversed for decryption.
This structure already highlights one main security issue, as the lack of an initial ``Add Key'' stage makes it so that the knowledge about the bits before the last ``Add Key'' automatically reveals the plaintext (the remaining stages do not contain `secrets' such as the key).

Furthermore, the key expansion procedure proposed for the simple AES can be seen to produce as subkeys $k_0$ and $k_1$ the first 8 bits and the last 8 bits respectively. For this reason, the key which is added during the last (and only) ``Add Key'' stage corresponds to the actual key itself, as provided by the user.

Figure \ref{fig:3.1} shows the structure of the described block cipher.

\begin{figure} [ht]
   \centering
   \includegraphics[width = .5\linewidth]{improper_block_scheme.jpeg}
   \caption{Structure of the improperly implemented block cipher}
   \label{fig:3.1}
\end{figure}

Putting all these things together, the proposed attack consists in performing the first two stages on the known plaintext (``Substitute Bytes'' and ``Shift Rows'') and then directly finding the key by evaluating the XOR between the result of the first stages and the known ciphertext. This last step guarantees the discovery of the key (in theory the ensemble of sub-keys $k_0$ and $k_1$), thanks to the properties of the XOR operator ($A\oplus B = C\ \implies A\oplus C = B$).

By performing these steps, the key used in the known pleintext-ciphertext pair was found to be \verb|0b100000111101111|.

The ciphertext was then decrypted, knowing the block cipher was used on two characters at a time (16 bits total).

The resulting plaintext contained a quote from `The Lord of the Rings':

\begin{quote}
   \textit{When Mr.\ Bilbo Baggins of Bag End announced that he would shortly be celebrating his eleventy-first birthday with a party of special magnificence, there was much talk and excitement in Hobbiton. 
   Bilbo was very rich and very peculiar, and had been the wonder of the Shire for sixty years, ever since his remarkable disappearance and unexpected return. 
   The riches he had brought back from his travels had now become a local legend, and it was popularly believed, whatever the old folk might say, that the Hill at Bag End was full of tunnels stuffed with treasure. 
   And if that was not enough for fame, there was also his prolonged vigour to marvel at. Time wore on, but it seemed to have little effect on Mr.\ Baggins.
   At ninety he was much the same as at fifty. 
   At ninety-nine they began to call him well-preserved, but unchanged would have been nearer the mark. 
   There were some that shook their heads and thought this was too much of a good thing; it seemed unfair that anyone should possess (apparently) perpetual youth as well as (reputedly) inexhaustible wealth}
\end{quote}

\subsection{Decryption under COA}
\label{sec:03.1}

As seen, breaking the encryption scheme under KPA is simple, as the mechanism lacks the initial XOR operation between plaintext and sub-key.
Supposing there is no available known plaintext–ciphertext pairs, the approach taken by an attacker needs to be different.

The overall encryption process for a long plaintext (such as the one in `ciphertext.txt'), then becomes similar to that of a Vigenère cipher, in which different portions (blocks) of the message are encrypted in the same way. In particular, after being passed through the S-Boxes and the ``Shift Rows'', the same key is applied on every set of 2 characters (16 bits).
\label{sec:01}

This laboratory revolved around the analysis of a simplified version of AES (Advanced Encryption Standard), which was provided as a python script, which can be found at \cite{Original Python implementation}, and which also contained all necessary sub–blocks for each round (SBox, ShiftRows, MixColumns and AddKey).
These blocks are designed to work on 16–bit inputs, organized as 2–by–2 matrices of 4–bit `\textit{nibbles}'. Also the key is a 16-bit string.

The focus of this analysis is to test the behavior of the simplified AES in terms of avalanche effect, and to cryptanalyze an improper implementation of a block cipher, based on the provided sub-blocks, in order to break a ciphertext.


\section{Conclusions}
\label{sec:04}

\begin{thebibliography}{5}
   \bibitem{Original Python implementation} \url{https://jhafranco.com/2012/02/11/simplified-aes-implementation-in-python}
\end{thebibliography}

\end{document}